% In this file you should put all LaTeX macros and settings to be used both by
% the pdf version and the web version.
% This should be most of your macros.

% The theorem-like environments defined below are those that appear by default
% in the dependency graph. See the README of leanblueprint if you need help to
% customize this.
% The configuration below use the theorem counter for all those environments
% (this is what the [theorem] arguments mean) and never resets it.
% If you want for instance to number them within chapters then you can add
% [chapter] at the end of the next line.
\newtheorem{theorem}{Théoreme}
\newtheorem{proposition}[theorem]{Proposition}
\newtheorem{lemma}[theorem]{Lemme}
\newtheorem{corollary}[theorem]{Corollaire}

\theoremstyle{definition}
\newtheorem{definition}[theorem]{Définition}

\theoremstyle{remark}
\newtheorem*{remark}{Remarque}

\addto\captionsfrench{\renewcommand\proofname{Preuve}}

% Math command

\newcommand{\R}{\mathbf R}
\newcommand{\C}{\mathbf C}
\newcommand{\N}{\mathbf N}
\newcommand{\Z}{\mathbf Z}
\newcommand{\Q}{\mathbf Q}
\newcommand{\K}{\mathbf K}
\newcommand{\T}{\mathbf T}
\renewcommand{\S}{\mathbf S}
\newcommand{\s}{\Sigma}
\renewcommand{\sp}{\Sigma^+}
\newcommand{\Sp}{\Sigma^+}
\newcommand{\Cs}[1]{\mathcal{C} (\Sigma_{#1})}
\newcommand{\Csp}[1]{\mathcal{C} (\Sigma_{#1}^+)}
\newcommand{\Hs}[1]{\mathcal{H} (\Sigma_{#1})}
\newcommand{\Hsp}[1]{\mathcal{H} (\Sigma_{#1}^+)}
\renewcommand{\L}{\mathcal{L}}
\newcommand{\Ms}[1]{\mathcal{M}(\Sigma_{#1})}
\newcommand{\Mss}[1]{\mathcal{M}_{\sigma}(\Sigma_{#1})}
\newcommand{\Msp}[1]{\mathcal{M} (\Sigma_{#1}^+)}
\renewcommand{\No}[1]{\left\lVert #1\right\rVert}
\newcommand{\Cr}[1]{\mathcal C_{#1}}
\newcommand{\ind}[1]{\mathbf{1}_{#1}}
\newcommand{\Bo}[1]{\mathcal{B}(#1)}
\newcommand{\Ws}[2][\varepsilon]{W^s_{#1}(#2)}
\newcommand{\Wu}[2][\varepsilon]{W^u_{#1}(#2)}

\renewcommand{\llbracket}{[\![}
\renewcommand{\rrbracket}{]\!]}

\DeclareMathOperator{\id}{id}
\DeclareMathOperator{\var}{var}
\DeclareMathOperator{\Spe}{Sp}
\DeclareMathOperator{\Card}{Card}
\DeclareMathOperator{\Int}{Int}
\DeclareMathOperator{\diam}{diam}
